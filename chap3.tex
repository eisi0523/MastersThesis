%% This is an example first chapter.  You should put chapter/appendix that you
%% write into a separate file, and add a line \include{yourfilename} to
%% main.tex, where `yourfilename.tex' is the name of the chapter/appendix file.
%% You can process specific files by typing their names in at the 
%% \files=
%% prompt when you run the file main.tex through LaTeX.

\chapter{LGR Performance and Field Characterization}

Whatever

\section{Quality Factor}

Qualtiy factor importance etc. Low Q importance for CW application (ie. fitting all resonances in for wide dynamic range magnetometer) and for pulsed (minimize ring down ie dead time between subsequent pulses)

Discussion on using concatinated pulses to mitigate long ring down as a method to use high Q resonators (like copper resonator) even though their ring down time is long. Attempt to work out the math on these pulses or give a small mathematical description of dead time mitigating pulses. Ie active cancellation

\subsection{Calculation}

calculating Q in different coupled regimes

\subsection{Measurement}

Measuring the Q using ring-down method and bandwidth method

\subsection{Ringdown time}

short mathematical walkthrough of calculating ring down time from Q factor. From Jackson preferably.


\section{Measuring Magnetic Field}

\subsection{Experimental setup}

talk about confocal microscope, give outline schematic of setup

\subsection{Measurement}

Talk about experiment taking Rabi data, ie. Rabi pulse sequence, moving resonator, checking ESR, choosing ESR etc.

\subsection{$B_1$ from Rabi}

calculating $B_1$ from Rabi frequency using rotating wave approx. etc, essentially where sqrt(3) comes from.





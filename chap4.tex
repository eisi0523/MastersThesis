%% This is an example first chapter.  You should put chapter/appendix that you
%% write into a separate file, and add a line \include{yourfilename} to
%% main.tex, where `yourfilename.tex' is the name of the chapter/appendix file.
%% You can process specific files by typing their names in at the 
%% \files=
%% prompt when you run the file main.tex through LaTeX.

\chapter{Discussion and Outlook} \label{ch4}

The device presented here exhibits further benefits which we now discuss, along with extensions tailored for specific applications. For example, for ubiquitously employed pulsed measurement protocols, a short ring-down time $\tau_\text{ring}$ (i.e., $B_1$ field $1/e$ decay time) is necessary for high-fidelity pulse shape control. Although techniques to compensate for long ring-down times are effective~\cite{tabuchi2010total,borneman2012bandwidth,peshkovsky2005rf}, shorter native values of $\tau_\text{ring}$ are nonetheless generally desired~\cite{pfenninger1995general,rinard2005loopgap}. The observed loaded quality factor $Q_L = 36$ corresponds to a ring-down time of $\tau_\text{ring} = 4$ ns (see section \ref{ringdown}), making the device suitable for standard pulsed protocols~\cite{Smeltzer2009Quantum, Jelezko2004Observation}. 

%This ring-down time is expected to allow pulsed MW protocols for high fidelity manipulation of the NV quantum state without deleterious ring-down artifacts.

Due to square-root scaling of $B_1$ with incident MW power ($B_1 \propto \sqrt{P}$), higher power handling can allow for stronger $B_1$ fields. The non-planar resonator design allows for otherwise higher incident MW powers as currents circulate over an extended 2D surface (versus the 1D edge for a planar structure). Further, the metallic LGR thermal mass and  thermal conductivity allow efficient heat transfer and sinking, resulting in improved device stability and power handling. Although the latter was not tested, the LGR composite device is expected to allow $>\!$ 100 W for CW and pulsed operation, limited by dielectric breakdown of air in the 260 $\upmu$m capacitive gaps. Should available MW power be constrained, stronger $B_1$ can be achieved by fabricating the LGR from a more electrically conductive material (e.g. silver or copper) at the expense of bandwidth (see section \ref{CopperLGR}). In such circumstances, the bandwidth can be continuously adjusted above its minimum value by over-coupling the resonator (at the expense of reduced $Q_L$). 

While the presented LGR is 5 mm thick, the fundamental hole-and-slot approach is expected to be feasible for a variety of thicknesses. A thicker device will provide better field uniformity at the expense of optical access. In contrast, for applications requiring MW delivery over a thin planar volume, we expect the LGR can be fabricated via deposition on an appropriate insulating substrate, as discussed in Refs.~\cite{twig2013ultra,twig2010sensitive}. We have found semi-insulating silicon carbide~\cite{schloss2018simultaneous} suitable due to the material's high thermal conductivity ($\approx$490 W/(m*K)~\cite{protik2017phonon,qian2017anisotropic}, high Young's modulus, moderate cost and wide availability in semi-conductor grade wafers. Our simulations suggest the planar LGR approach can offer modest improvements in $B_1$ homogeneity over split ring resonators. 

%\textcolor{red}{[Somebody (i.e., Roberto) please confirm the veracity of the last statement.]}

Although the exciter antenna (see Section \ref{excitation}) facilitates a compact, vibration-resistant, and portable device, this component introduces non-idealities in both field uniformity and optical access. As similar scattering parameters are obtained by inductively coupling a small coil to one of the LGR outer loops, this latter solution may find favor for applications requiring maximal optical access and, furthermore, requires no PCB fabrication.


In this work, we demonstrated a broadband tunable LGR allowing appplication of strong homogeneous MW fields to an NV ensemble. The LGR demonstrates a dramatic improvement over prior MW delivery mechanisms, both improving on and spatially extending MW field homogeneities. We expect the device to be useful for bulk sensing~\cite{acosta2009diamonds,wolf2015subpicotesla,clevenson2015broadband,chatzidrosos2017miniature,barry2016optical} and particularly imaging applications~\cite{karaveli2016modulation,glenn2015single,barry2016optical,lesage2013optical,wu2016diamond,fu2014solar,glenn2017micrometer}, due to the optical access allowed by the LGR composite device both above and below the diamond.





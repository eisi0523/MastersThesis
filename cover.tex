% -*-latex-*-
% 
% For questions, comments, concerns or complaints:
% thesis@mit.edu
% 
%
% $Log: cover.tex,v $
% Revision 1.8  2008/05/13 15:02:15  jdreed
% Degree month is June, not May.  Added note about prevdegrees.
% Arthur Smith's title updated
%
% Revision 1.7  2001/02/08 18:53:16  boojum
% changed some \newpages to \cleardoublepages
%
% Revision 1.6  1999/10/21 14:49:31  boojum
% changed comment referring to documentstyle
%
% Revision 1.5  1999/10/21 14:39:04  boojum
% *** empty log message ***
%
% Revision 1.4  1997/04/18  17:54:10  othomas
% added page numbers on abstract and cover, and made 1 abstract
% page the default rather than 2.  (anne hunter tells me this
% is the new institute standard.)
%
% Revision 1.4  1997/04/18  17:54:10  othomas
% added page numbers on abstract and cover, and made 1 abstract
% page the default rather than 2.  (anne hunter tells me this
% is the new institute standard.)
%
% Revision 1.3  93/05/17  17:06:29  starflt
% Added acknowledgements section (suggested by tompalka)
% 
% Revision 1.2  92/04/22  13:13:13  epeisach
% Fixes for 1991 course 6 requirements
% Phrase "and to grant others the right to do so" has been added to 
% permission clause
% Second copy of abstract is not counted as separate pages so numbering works
% out
% 
% Revision 1.1  92/04/22  13:08:20  epeisach

% NOTE:
% These templates make an effort to conform to the MIT Thesis specifications,
% however the specifications can change.  We recommend that you verify the
% layout of your title page with your thesis advisor and/or the MIT 
% Libraries before printing your final copy.
\title{Tunable and broadband loop gap resonator for nitrogen vacancy centers in diamond}

\author{Erik Roger Eisenach}
% If you wish to list your previous degrees on the cover page, use the 
% previous degrees command:
%       \prevdegrees{A.A., Harvard University (1985)}
% You can use the \\ command to list multiple previous degrees
%       \prevdegrees{B.S., University of California (1978) \\
%                    S.M., Massachusetts Institute of Technology (1981)}
\prevdegrees{B.S., The Citadel, The Military College of South Carolina (2015)}
\department{Department of Electrical Engineering and Computer Science}

% If the thesis is for two degrees simultaneously, list them both
% separated by \and like this:
% \degree{Doctor of Philosophy \and Master of Science}
\degree{Master of Science in Electrical Engineering and Computer Science}

% As of the 2007-08 academic year, valid degree months are September, 
% February, or June.  The default is June.
\degreemonth{June}
\degreeyear{2018}
\thesisdate{May 23, 2018}

%% By default, the thesis will be copyrighted to MIT.  If you need to copyright
%% the thesis to yourself, just specify the `vi' documentclass option.  If for
%% some reason you want to exactly specify the copyright notice text, you can
%% use the \copyrightnoticetext command.  
%\copyrightnoticetext{\copyright IBM, 1990.  Do not open till Xmas.}

% If there is more than one supervisor, use the \supervisor command
% once for each.
\supervisor{Dirk Englund}{Associate Professor, Massachusetts Institute of Technology}
\supervisor{Danielle Braje}{Assistant Group Leader, Lincoln Laboratory}

% This is the department committee chairman, not the thesis committee
% chairman.  You should replace this with your Department's Committee
% Chairman.
\chairman{Leslie A. Kolodziejski}{Chairman, Department Committee on Graduate Theses}

% Make the titlepage based on the above information.  If you need
% something special and can't use the standard form, you can specify
% the exact text of the titlepage yourself.  Put it in a titlepage
% environment and leave blank lines where you want vertical space.
% The spaces will be adjusted to fill the entire page.  The dotted
% lines for the signatures are made with the \signature command.
\maketitle

% The abstractpage environment sets up everything on the page except
% the text itself.  The title and other header material are put at the
% top of the page, and the supervisors are listed at the bottom.  A
% new page is begun both before and after.  Of course, an abstract may
% be more than one page itself.  If you need more control over the
% format of the page, you can use the abstract environment, which puts
% the word "Abstract" at the beginning and single spaces its text.

%% You can either \input (*not* \include) your abstract file, or you can put
%% the text of the abstract directly between the \begin{abstractpage} and
%% \end{abstractpage} commands.

% First copy: start a new page, and save the page number.
\cleardoublepage
% Uncomment the next line if you do NOT want a page number on your
% abstract and acknowledgments pages.
% \pagestyle{empty}
\setcounter{savepage}{\thepage}
\begin{abstractpage}
% $Log: abstract.tex,v $
% Revision 1.1  93/05/14  14:56:25  starflt
% Initial revision
% 
% Revision 1.1  90/05/04  10:41:01  lwvanels
% Initial revision
% 
%
%% The text of your abstract and nothing else (other than comments) goes here.
%% It will be single-spaced and the rest of the text that is supposed to go on
%% the abstract page will be generated by the abstractpage environment.  This
%% file should be \input (not \include 'd) from cover.tex.



Nitrogen vacancy centers in diamond have emerged as a solid-state analog to atomic systems with applications ranging from room temperature quantum computing to quantum sensing and metrology. To date, with notably few exceptions, all NV applications rely on coherent manipulation of spin states via resonant microwave driving. In this thesis the loop gap resonator (LGR) is presented as a mechanism for the delivery of resonantly enhanced and uniform microwave fields to large volume samples of nitrogen vacancy (NV) centers in diamond. Specifically, an S-band tunable LGR and its constituent excitation circuitry are designed and fabricated to enable directionally uniform, strong, homogeneous, and broadband
microwave (MW) driving of an NV ensemble over an area larger than 32 mm\textsuperscript{2}. The LGR design, based on the anode
block of a cavity magnetron, demonstrates an average field amplitude of 5 gauss at 42 dBm of input power, and achieves a peak-to-peak
field uniformity of 89.5\% over an area of 32 mm\textsuperscript{2} and 97\% over an area of 11 mm\textsuperscript{2}. The broad bandwidth of the LGR is capable of addressing all resonances of an NV ensemble for bias magnetic Fields up to 14 gauss. Furthermore, with cavity
ring-down-times in the single nanoseconds, the resonator is compatible with the pulsed MW techniques necessary for a wide range of NV-diamond
applications.

%Need to chat about copper resonator and smaller 5mm resonator
\end{abstractpage}

% Additional copy: start a new page, and reset the page number.  This way,
% the second copy of the abstract is not counted as separate pages.
% Uncomment the next 6 lines if you need two copies of the abstract
% page.
% \setcounter{page}{\thesavepage}
% \begin{abstractpage}
% % $Log: abstract.tex,v $
% Revision 1.1  93/05/14  14:56:25  starflt
% Initial revision
% 
% Revision 1.1  90/05/04  10:41:01  lwvanels
% Initial revision
% 
%
%% The text of your abstract and nothing else (other than comments) goes here.
%% It will be single-spaced and the rest of the text that is supposed to go on
%% the abstract page will be generated by the abstractpage environment.  This
%% file should be \input (not \include 'd) from cover.tex.



Nitrogen vacancy centers in diamond have emerged as a solid-state analog to atomic systems with applications ranging from room temperature quantum computing to quantum sensing and metrology. To date, with notably few exceptions, all NV applications rely on coherent manipulation of spin states via resonant microwave driving. In this thesis the loop gap resonator (LGR) is presented as a mechanism for the delivery of resonantly enhanced and uniform microwave fields to large volume samples of nitrogen vacancy (NV) centers in diamond. Specifically, an S-band tunable LGR and its constituent excitation circuitry are designed and fabricated to enable directionally uniform, strong, homogeneous, and broadband
microwave (MW) driving of an NV ensemble over an area larger than 32 mm\textsuperscript{2}. The LGR design, based on the anode
block of a cavity magnetron, demonstrates an average field amplitude of 5 gauss at 42 dBm of input power, and achieves a peak-to-peak
field uniformity of 89.5\% over an area of 32 mm\textsuperscript{2} and 97\% over an area of 11 mm\textsuperscript{2}. The broad bandwidth of the LGR is capable of addressing all resonances of an NV ensemble for bias magnetic Fields up to 14 gauss. Furthermore, with cavity
ring-down-times in the single nanoseconds, the resonator is compatible with the pulsed MW techniques necessary for a wide range of NV-diamond
applications.

%Need to chat about copper resonator and smaller 5mm resonator
% \end{abstractpage}

\cleardoublepage

\section*{Acknowledgments}

When I came to MIT I had the great fortune to begin my graduate work under two amazing advisors, Dirk Englund and Danielle Braje. I am indebted to them for all the work they put into mentoring and shaping me into the scientist and engineer that I am today. Their advice has been invaluable to me both personally and in the successful completion of this work. In conjunction, I want to acknowledge all my friends and colleagues in both the Quantum Photonics Lab at MIT and the Quantum Sensing Lab at Lincoln. In particular, I want to thank both John Barry and Linh Pham who work tirelessly in the lab and played an integral role in generating the work presented in this text. I have learned an incredible amount from the two of them and am looking forward to learning more in the years to come. Additionally, I want to thank both Chris McNally and Scott Alsid who kept me thoroughly entertained with many interesting conversations about physics and other topics. I wish them both all the best in their future endeavors. I am also grateful to Hannah Clevenson and Ed Chen for their friendship and their willingness to share advice on managing the difficulties of graduate school. Back at the Citadel in South Carolina, I want to thank all my excellent ex-professors in the departments of Electrical Engineering and Physics who provided a solid foundation of knowledge on which I could build. I want to extend a special thank you to Gregory Mazzaro who--through many wonderful conversations--placed me on the PhD track in the first place. I cherish his friendship and the emails we have shared over the past couple years. 

Outside of the laboratory, I want to thank Marvin and Susan Krause whose love and support keep me going every day. They are, and have been for much of my adult life, an unwavering source of support and happiness. I want to thank my father Hans Eisenach and mother Paula Eisenach for their love and their hard work of shaping me into the man I am today. I miss them both every day, especially my late mom who, I am sure, knows how far I have come. Most of all I want to thank my beautiful, smart and supportive wife, Monique Eisenach, without whom I would have given up many years ago. She gives me strength every-day to work hard and not look back on the mistakes I have made in the past. She makes me laugh and for a moment forget how painful graduate school can be. Her love and friendship is the fuel that powers everything I am and everything I do.


%%%%%%%%%%%%%%%%%%%%%%%%%%%%%%%%%%%%%%%%%%%%%%%%%%%%%%%%%%%%%%%%%%%%%%
% -*-latex-*-

% $Log: abstract.tex,v $
% Revision 1.1  93/05/14  14:56:25  starflt
% Initial revision
% 
% Revision 1.1  90/05/04  10:41:01  lwvanels
% Initial revision
% 
%
%% The text of your abstract and nothing else (other than comments) goes here.
%% It will be single-spaced and the rest of the text that is supposed to go on
%% the abstract page will be generated by the abstractpage environment.  This
%% file should be \input (not \include 'd) from cover.tex.



Nitrogen vacancy centers in diamond have emerged as a solid-state analog to atomic systems with applications ranging from room temperature quantum computing to quantum sensing and metrology. To date, with notably few exceptions, all NV applications rely on coherent manipulation of spin states via resonant microwave driving. In this thesis the loop gap resonator (LGR) is presented as a mechanism for the delivery of resonantly enhanced and uniform microwave fields to large volume samples of nitrogen vacancy (NV) centers in diamond. Specifically, an S-band tunable LGR and its constituent excitation circuitry are designed and fabricated to enable directionally uniform, strong, homogeneous, and broadband
microwave (MW) driving of an NV ensemble over an area larger than 32 mm\textsuperscript{2}. The LGR design, based on the anode
block of a cavity magnetron, demonstrates an average field amplitude of 5 gauss at 42 dBm of input power, and achieves a peak-to-peak
field uniformity of 89.5\% over an area of 32 mm\textsuperscript{2} and 97\% over an area of 11 mm\textsuperscript{2}. The broad bandwidth of the LGR is capable of addressing all resonances of an NV ensemble for bias magnetic Fields up to 14 gauss. Furthermore, with cavity
ring-down-times in the single nanoseconds, the resonator is compatible with the pulsed MW techniques necessary for a wide range of NV-diamond
applications.

%Need to chat about copper resonator and smaller 5mm resonator